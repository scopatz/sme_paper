\documentclass[12pt]{article}

\usepackage{fancyhdr}

\pagestyle{fancy}

\fancyhead[L]{Anthony Scopatz}
\fancyhead[R]{\today}
\fancyfoot[L]{Novelty Statement}
\fancyfoot[C]{}
\fancyfoot[R]{\thepage}

\begin{document}
\begin{center}
\section*{Novelty Statement}
\end{center}

The work presented in submission entitled ``Symbolic Multicomponent Enrichment for 
Matched Abundance Ratio Cascades'' is new and novel primarily because it takes 
a radical approach to solving the matched abundance ratio cascade (MARC) equations.  
This system of equations has been known for decades, since de la Garza in the 1960s.
Typically, they are solved using numerical optimization techniques.  Some work
has even gone into the characterization of these different solvers (Song, et al.).

Here, a symbolic mathematics library (SymPy) was used to solve this system
of equations by variable substitution and elimination.  Since some independent 
variables cannot be expressed exactly in a closed form way, as symbolic approximation
to the closed form solution was used.  Due to the number of terms in this
approximation a symbolic math library was needed to handle the resultant expression.
This is why the representation of the MARC system used here was not previously
discovered.

Furthermore, while symbolic solutions are typically quite slow, the solver here 
was coupled with a custom C-code generator.  This enables the cascade solutions to be 
computed 20 - 300 times faster than other nominal numerical optimizers. 
Additionally, it is shown that the solution provided this symbolic solver is 
just as accurate as any numeric solver and matches results published by many 
other groups.

Thus this work represents a new, significantly faster mechanism for computing the 
MARC equations using novel closed form representations of many of the independent 
variables.  Furthermore, the innovative symbolic to machine code pipeline that 
enables this solution was created solely to aid in this calculation.

\end{document}
